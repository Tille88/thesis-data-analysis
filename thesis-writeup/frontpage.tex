
   
\includepdf[fitpaper=true, pages=-]{pdf/FrontPageGISthesis.pdf}


\pagenumbering{roman}
\setcounter{page}{2}

% TODO: ENSURE THIS PAGE START NUMBERING AS (ii)

\textbf{Jonas Tillman (2021). Legend design perception effects and user acceptance for opacity data mapping over GIS backgrounds}

Master degree thesis, 30 credits in Geographical Information Systems (GIS) 

Department of Physical Geography and Ecosystem Science, Lund University

\newpage



\begin{center}
    \vspace*{1cm}
        
    \Huge
    \textbf{Legend design perception effects and user acceptance for opacity data mapping over GIS backgrounds}
        
        
    \vspace{5cm}
        
    \rule{15cm}{1.0pt}

    \Large
    Jonas Tillman

    Master thesis, 30 credits, in Geographical Information Sciences
    
    \vspace{2cm}
    
    Supervisor: Micael Runnström PhD

    GIS Centre
    
    Dept of Physical Geography and Ecosystem Science
    
    Lund University
                    
\end{center}

\newpage

\abstract{
\noindent Data visualisation and GIS have clear overlapping goals in ensuring that the users can decode visual data encodings to understand viewed data. This study looks at visual encoding using opacity/transparency of data encoded using the alpha channel and overlaid on a map background. The focus is on if different legend design types support in decoding of the data in the opacity layer, and the objective reception of the different legend types among the respondents. 
The study design presented 5 different legend types to respondents, with the intention of the only non-random variation being the legend types. Baseline categories were (I) no legend, only having the range of the data values being expressed in text and (II) a legend design imitating the ArcGIS default legend for opacity data mapping.

\vspace{0.4cm}

\noindent The remaining legend designs attempted to introduce more contextualisation from the map background to the background of the legend, and reducing the distance from the legend to the data. This was done using (I) a sample of the map, (II) having the most prevalent colours of the map as background through running a clustering algorithm over the map background data and (III) attaching a legend to the data area - which both would add contextualisation since the background is the map, and reducing distance to the data.

\vspace{0.4cm}

\noindent A back-end and front-end system was developed to choose progressions of examples to respondents in a semi-random manner, and the users were requested to visually estimate the value of the location of a marker within the data area.

\vspace{0.4cm}

\noindent In statistical analysis of the results there was a clear statistical effects in reduced errors when having a legend compared to when no legend was included. There were, however, no statistically significant difference in estimation/perception errors between legend types. 

\vspace{0.4cm}

\noindent The acceptance of respondents as to how useful they considered the legend types did have statistically higher acceptance estimates compared to the default ArcGIS design for when sampling the map background and when attaching the legend to the data area. These results indicates that there may be reasons for considering alternative legend design to create a better user experience when having complex GIS data visualisations.

\vspace{0.4cm}

\noindent \textbf{Keywords:} Geography, GIS, Data Visualisation, Legends, Opacity, Transparency, Visual Data Encoding, Perception
}

% \includepdf[fitpaper=true, pages=-, pagecommand={}]{pdf/abstract_scientific_summary.pdf}

