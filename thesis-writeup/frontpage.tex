
   
% \includepdf[fitpaper=true, pages=-]{pdf/FrontPageGISthesis.pdf}


\pagenumbering{roman}
\setcounter{page}{2}

% TODO: ENSURE THIS PAGE START NUMBERING AS (ii)

\textbf{Jonas Tillman (2021). Perception accuracy and user acceptance of legend design choices}

\emph{A study of legend designs for opacity data mapping in GIS}

Master degree thesis, 30 credits in Geographical Information Systems (GIS) 

Department of Physical Geography and Ecosystem Science, Lund University

\newpage



\begin{center}
    \vspace*{1cm}
        
    \Huge
    \textbf{Perception accuracy and user acceptance of legend design choices}
    
    \Large
    \emph{A study of legend designs for opacity data mapping in GIS}
        
    \vspace{5cm}
        
    \rule{15cm}{1.0pt}

    \Large
    Jonas Tillman

    Master thesis, 30 credits, in Geographical Information Sciences
    
    \vspace{2cm}
    
    Supervisor: Micael Runnström PhD

    GIS Centre
    
    Dept of Physical Geography and Ecosystem Science
    
    Lund University
                    
\end{center}

\newpage

\abstract{
\noindent In a GIS system, the need to encode non-cartographic geospatial data on top of a map is common. This can be done with an overlay layer on top of a base map, with the effect that the base map is partially or fully hidden. Reducing the opacity (or its equivalent - increasing the transparency) of this overlay data layer is a frequently seen solution to show the geographic context. With the colours of the base map being combined with the overlay layer's colours, the resulting visualization can become difficult to interpret for the end-user.

\vspace{0.4cm}

\noindent To help the user decode the values encoded in maps, legends are a common tool for non-interactive maps and data visualizations. This study investigates the decoding accuracy for 5 different legend designs for opacity-mapped data overlayed on a static base map. A secondary objective is to measure the acceptance of users for these different legend designs.

\vspace{0.4cm}

\noindent The study design presented 5 different legend types to respondents. Baseline categories for comparison were (i) no legend - only having the range of the data values being expressed in text and (ii) a legend design imitating the ArcGIS legend for opacity data mapping.

\vspace{0.4cm}

\noindent The remaining legend designs attempted to introduce more contextualisation from the map background to the background of the legend and reducing the distance from the legend to the data. This was done using (a) a sample of the map as background for the legend (b) having the most prevalent colours of the map as the legend background (c) attaching a legend directly to the edge of the overlay data area.

\vspace{0.4cm}

\noindent Using a web interface, the users were requested to visually estimate the value at the location of a marker within the overlay data area. In statistical analysis of the results, there was clear statistical effect in reduced errors when having a legend compared to when no legend was included. There was, however, no statistically significant difference in estimation/perception errors between legend designs. 

\vspace{0.4cm}

\noindent The acceptance of respondents - defined as how useful they considered the legend types were to help guess the value - did have statistically higher estimates when sampling the map background (a) and when attaching the legend to the data area (c) compared to the default ArcGIS design (i). 

\vspace{0.4cm}

\noindent \textbf{Keywords:} GIS, Data Visualisation, Legends, Opacity, Transparency, Visual Data Encoding, Perception
}

% \includepdf[fitpaper=true, pages=-, pagecommand={}]{pdf/abstract_scientific_summary.pdf}

